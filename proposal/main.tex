\documentclass{article}

% ---------------------------------------------------------------------------
% PREAMPLE
% ---------------------------------------------------------------------------
\usepackage{arxiv}
\usepackage[utf8]{inputenc} % allow utf-8 input
\usepackage[T1]{fontenc}    % use 8-bit T1 fonts
\usepackage{hyperref}       % hyperlinks
\usepackage{url}            % simple URL typesetting
\usepackage{booktabs}       % professional-quality tables
\usepackage{amsfonts}       % blackboard math symbols
\usepackage{nicefrac}       % compact symbols for 1/2, etc.
\usepackage{microtype}      % microtypography
\usepackage{cleveref}       % smart cross-referencing
\usepackage{lipsum}         % Can be removed after putting your text content
\usepackage{graphicx}
\usepackage{natbib}
\usepackage{doi}

% ---------------------------------------------------------------------------
% TITLE DEFINITION
% ---------------------------------------------------------------------------
\title{\emph{Project proposal}: Experimentally improving a YOLO-inspired object detection model}

\author{
	Herluf Baggesen \\
	201800258@post.au.dk \\
	\And
	Mads Buchmann Frederiksen \\
	202003444@post.au.dk\\
}

\date{}
\renewcommand{\undertitle}{}
\renewcommand{\shorttitle}{Improving object detection model}

\hypersetup{
	pdftitle={Project proposal: Experimentally improving a YOLO-inspired object detection model},
	pdfsubject={},
	pdfauthor={Herluf Baggesen, Mads Buchmann Frederiksen},
	pdfkeywords={},
}

% ---------------------------------------------------------------------------
% DOCUMENT
% ---------------------------------------------------------------------------
\begin{document}
\maketitle

For our course project we propose to take an experimental approach to improving a base models performance on the task of object detection. We plan to employ the simple model presented in \href{https://github.com/klaverhenrik/Deep-Learing-for-Visual-Recognition-2022/blob/main/Lab3_FunWithMNIST.ipynb}{Lab 3}, which is a naive implementation inspired by the YOLO architecture \citep{YOLO}, as our base model. The overall purpose of the project is to investigate and understand the effect of certain modifications to the model and training procedure through experimentation. 

\subheading{Task & Dataset}
For the object detection task we have chosen digit detection using the \href{http://ufldl.stanford.edu/housenumbers/}{Street View House Number Dataset (SVHN)}. The main advantages of this dataset is that it is easily available, of good quality and of reasonable size, while being small enough to feasibly train models on the hardware available to us.

At the time of writing we plan to experiment with:
\begin{itemize}
\item Using different loss functions. This is motivated by the fact that object detection consist of both regression, when finding a bounding box, and classification. The naive model uses only mean squared error loss, even though it is unsuited for classification tasks. We expect that this change will make the model converge faster during training.
\item Adding depth to both the encoding and decoding steps. We expect to find that adding layers with some motivated combination of activations will improve the models performance dramatically.
\item Adding skip connections. This is a known technique to address the issue of "exploding gradients" that we would like to investigate. This item is motivated by \citep{SGP}.
\end{itemize}

We expect, as the project and course progresses, that we should identify more techniques that will motivate further experimentation. We leave some room in our planning to pursue these angles as well. 

To compare the base model with a modified model during a particular experiment, we plan to:
\begin{itemize}
	\item Compute intersection over union to measure bounding box accuracy.
	\item Compute mean average precision for classification of digits.
	\item Investigate precision/recall curves
	\item Investigate confusion matrices to detect misclassification trends
\end{itemize}

We plan to implement our project using the \href{https://pytorch.org/}{pytorch} API, as it allows for the granular level of tweaking that we want to pursue. We plan to begin working on these steps first:
\begin{enumerate}
	\item Implementing and training the base model in pytorch
	\item Implement a reusable way to investigate model performance via the metrics mentioned earlier. This will allow for faster experimentation.
	\item Begin performing an experiment, presumably with the loss function. 
\end{enumerate}

% ---------------------------------------------------------------------------
% BIBLOGRAPHY
% ---------------------------------------------------------------------------
\bibliographystyle{plain}
\bibliography{references}

\end{document}
